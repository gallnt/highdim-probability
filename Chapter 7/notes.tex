\section{Random Processes}
This chapter concerns mostly with random processes - collection random variables $(X_t)_{t \in T}$, which 
may be dependent. In calssical settings like Brownian motion, $t$ represents time so $T \subset \mathbb{R}$. 
However, in high-dimensional probability $T$ can be any set, and we'll deal with Gaussian processes a lot.

In this chapter, we'll explore powerful comparison inequalities for Gaussian processes - Slepian, 
Sudakov-Frenique, and Gordon - by using a new trick: Gaussian interpolation. Then we use these tools to prove 
a sharp bound on the operator norm of $m \times n$ Gaussian random matrices.

How does a Gaussian process $(X_t)_{t \in T}$ capture the geometry of $T$? We'll prove a lower bound on the 
Gaussian width using covering numbers, and link it to other ideas like effective dimension. Moreover, we'll 
also compute the size of a ranodm projection of any bounded set $T \subset \mathbb{R}^n$, which heavily depends 
on the Gaussian width.



% ----------7.1----------
\subsection{Basic Concepts and Examples}



% ----------7.2----------
\subsection{Slepian, Sudakov-Frenique, and Gordon Inequalities}



% ----------7.3----------
\subsection{Application: Sharp Bounds for Gaussian Matrices}



% ----------7.4----------
\subsection{Sudakov Inequality}



% ----------7.5----------
\subsection{Gaussian Width}



% ----------7.6----------
\subsection{Application: Random Projection of Sets}

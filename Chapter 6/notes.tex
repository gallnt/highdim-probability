\section{Quadratic Forms, Symmetrization, and Contraction}
This section concerns mostly with decoupling, concentration of quadratic forms, symmetrization, and contraction, 
which are a number of basic toold of high-dimensional probability.



% ----------6.1----------
\subsection{Decoupling}
We'll look at \underline{quadratic forms} of the form
\[ \sum_{i, j = 1}^{n} a_{ij} X_i X_j = X^T AX = \left\langle X, AX \right\rangle \]
where $A = (a_{ij})$ is an $n \times n$ coefficient matrix and $X = (X_1, \dots, X_n)$ is a random vector 
with independent coordinates. Such quadratic forms are known as \underline{chaos}.

We can compute the expectation of a chaos. If $X_i$ have zero means and unit variances, then 
\[ \mathbb{E}[X^T AX] = \sum_{i, j = 1}^{n} a_{ij} \mathbb{E}[X_i X_j] 
= \sum_{i = 1}^{n} a_{ii} = \mathrm{tr}(A). \]

However, establishing concentration on a chaos is harder, because the terms of the sum above are not 
independent. However, we can overcome this difficulty via \underline{decoupling}. We'll replace the quadratic 
form above with the bilinear form 
\[ \sum_{i, j = 1}^{n} a_{ij}  X_iX_j' = X^T AX' = \left\langle X, AX' \right\rangle,  \]
where $X' = (X_1', \dots, X_n')$ is an independent copy of $X$. Bilinear forms are easier to analyze than 
quadratic forms as they are linear in $X$. Therefore if we condition on $X'$, we may treat the bilinear form 
as a sum of independent random variables
\[ \sum_{i = 1}^{n} \left( \sum_{j = 1}^{n} a_{ij} X_j' \right) X_i = \sum_{i = 1}^{n} b_i X_i \]
with fixed coefficients $b_i$.

\begin{theorem}[Decoupling]
\label{thm:6.1.1}
Let $A$ be an $n \times n$ diagonal free matrix, i.e. all diagonal entries are zero. Let $X$ be a random vector 
in $\mathbb{R}^n$ with independent mean zero coordinates, and let $X'$ be an independent copy. Then for every 
convex function $F: \mathbb{R} \to \mathbb{R}$, 
\[ \mathbb{E}[F(X^T AX)] \leq \mathbb{E}[F(4X^T AX')]. \]
\end{theorem}

\begin{proof}
We'll replace the chaos by a partial chaos, which we extend back to the original chaos later via Jensen's 
inequality. The partial chaos is defined by 
\[ \sum_{(i, j) \in I \times I^c}^{} a_{ij}X_iX_j \]
where $I \subset \{1, \dots, n\}$ is a randomly chosen subset of indices.

(\textbf{Step 1: Randomly selecting a partial sum}) To specify a random subset of indices $I$, we'll use 
\underline{selectors} - independent Bernoulli random variables $\delta_1, \dots, \delta_n \sim_{iid} 
\mathrm{Ber}(1/2)$. We define the index set 
\[ I := \{ i: \delta_i = 1 \}. \]
Condition on $X$. Since by assumption $a_{ii} = 0$ and 
\[ \mathbb{E}[\delta_i(1 - \delta_j)] = \frac{1}{2} \cdot \frac{1}{2} = \frac{1}{4} \text{ for all } i \neq j \]
we may express the chaos as 
\[ X^T AX = \sum_{i \neq j}^{} a_{ij} X_iX_j 
= 4 \mathbb{E}_{\delta}\left[ \sum_{i \neq j}^{} \delta_i(1 - \delta_j) a_{ij} X_iX_j \right] 
= 4 \mathbb{E}_I \left[ \sum_{(i, j_ \in I \times I^C)}^{} a_{ij}X_iX_j \right]. \]
(In the expression above, the subscripts $\delta$ and $I$ indicate the source of randomness in the conditional 
expectations. Since $X$ is fixed, the expectations are taken over the random selection of 
$\delta = (\delta_1, \dots, \delta_n)$, or equivalently, the random index set $I$).

(\textbf{Step 2: Applying $F$}) Applying the function $F$ to both sides and take expectation over $X$. By 
Jensen inequality and Fubini theorem, we get 
\[ \mathbb{E}_X[F(X^T AX)] \leq \mathbb{E}_I \left[ \mathbb{E}_X \left[ 
F \left( 4 \sum_{(i, j) \in I \times I^c}^{} a_{ij}X_iX_j \right) \right] \right]. \]
It follows that there exists a realization of a subset $I$ such that 
\[ \mathbb{E}_X[F(X^T AX)] \leq \mathbb{E}_X \left[ 
F \left( 4 \sum_{(i, j) \in I \times I^c}^{} a_{ij}X_iX_j \right) \right]. \]
Fix such a realization $I$ until the end of the proof, and drop the subscript $X$ on the expectation for 
convenience. Since the random variables $(X_i){i \in I}$ are independent from $(X_j)_{j \in I^c}$, the 
distribution of the sum in the right side will not change if we replace $X_j$ by $X_j'$ hence
\[ \mathbb{E}_X[F(X^T AX)] \leq \mathbb{E} \left[ 
F \left( 4 \sum_{(i, j) \in I \times I^c}^{} a_{ij}X_iX_j' \right) \right]. \]

(\textbf{Step 3: Completing the partial sum}) It remains to complete the sum on the RHS to the sum over all 
pairs of indices. We want to show that 
\[ \mathbb{E} \left[ F \left( 4 \sum_{(i, j) \in I \times I^c}^{} a_{ij}X_iX_j' \right) \right] \leq 
\mathbb{E} \left[ F \left( 4 \sum_{(i, j) \in [n] \times [n]}^{} a_{ij}X_iX_j' \right) \right] \]
where $[n] = \{1, \dots, n\}$. To do this, we can decompose the sum on the right side as 
\[ \sum_{(i, j) \in [n] \times [n]}^{} a_{ij}X_iX_j' 
= \underbrace{ \sum_{(i, j) \in I \times I^c}^{} a_{ij}X_iX_j' }_{Y} 
+ \underbrace{\sum_{(i, j) \in I \times I}^{} a_{ij}X_iX_j' 
+ \sum_{(i, j) \in I^c \times [n]}^{} a_{ij}X_iX_j'}_{Z}\]
Condition on all $(X_i)_{i \in I}$ and $(X_j')_{j \in I^c}$, and denote this expectation by $\mathbb{E}'$. 
This fixes $Y$, while $Z$ has zero conditional expectation (check). Thus, by Jensen inequality, we get 
\[ F(4Y) = F(4Y + \mathbb{E}'[4Z]) = F(\mathbb{E}'[4Y + 4Z]) \leq \mathbb{E}'[F(4Y + 4Z)]. \]
Finally, taking expectations over all remaining random variables, we get 
\[ \mathbb{E}[F(4Y)] \leq \mathbb{E}[F(4Y + 4Z)]. \]
Hence the proof is complete.
\end{proof}

\begin{remark}[Diagonal-free assumption]
\label{rmk:6.1.2}
The assumption is essential in \cref{thm:6.1.1}, since the conclusion fails for diagonal matrices when 
$F(x) = x$. But we can include the diagonal on the right hand side: for any $n \times n$ matrix $A = (a_{ij})$, 
we get 
\[ \mathbb{E}\left[ F \left( \sum_{i \neq j}^{} a_{ij}X_iX_j \right) \right] 
\leq \mathbb{E}\left[ F \left( 4 \sum_{i, j}^{} a_{ij}X_iX_j' \right) \right] \]
This is shown in Exercise 6.1, and there are other variants of decoupling (Exercises 6.2-6.4).
\end{remark}



% ----------6.2----------
\subsection{Hanson-Wright Inequality}
If $X$ is a subgaussian random vector in $\mathbb{R}^n$, what can we say about its norm? If $X$ has 
indepdendent entries, then it concentrated (\cref{thm:3.1.1}). But in general, it does not have to - it can be 
too small with high probability (Exercise 3.37). However, it can't be too large:

\begin{proposition}[Norm of subgaussian random vector]
\label{prop:6.2.1}
Let $X$ be a mean zero subgaussian random vector in $\mathbb{R}^n$ with $\lVert X \rVert_{\psi_2} \leq K$. Then 
for every $t \geq 0$, 
\[ P(\lVert X \rVert_{2} \geq CK(\sqrt{n} + t)) \leq e^{-t^2}. \]
\end{proposition}

\begin{proof}
WLOG, we can assume that $K = 1$. Squaring and exponentiating both sides and using Markov's inequality, we get 
\[ P(c \lVert X \rVert_{2} \geq \sqrt{n} + t) \leq e^{-(n + t^2)} \mathbb{E}\left[ \exp{(c^2 
\lVert X \rVert_{2}^2 )} \right]. \]
Now we will use a \textbf{Gaussian replacement} trick: for some absolute constant $c > 0$, we claim that 
\[ \mathbb{E}\left[ \exp{(c^2 \lVert X \rVert_{2}^2)} \right] \leq 
\mathbb{E}\left[ \exp{(\lVert g \rVert_{2}^2 / 6)} \right] \text{ where } g \sim N(0, I_n). \]
To see this, condition on $X$ (treating it as a fixed vector); then $\left\langle g, X \right\rangle \sim 
N(0, \lVert X \rVert_{2}^2)$ by \cref{cor:3.3.2}, so (using the MGF of a normal distribution)
\[ \mathbb{E}\left[ \exp{(c^2 \lVert X \rVert_{2}^2)} \right] = 
\mathbb{E}_g\left[ \exp{(\sqrt{2}c \left\langle g, X \right\rangle )} \right], \]
where $\mathbb{E}_g$ denotes the conditional expectation over $g$. Now takes expectation over $X$ over both 
sides and apply Fubini:
\[ \mathbb{E}_X\left[ \exp{(c^2 \lVert X \rVert_{2}^2)} \right] 
= \mathbb{E}_X\left[ \mathbb{E}_g\left[ \exp{(\sqrt{2}c \left\langle g, X \right\rangle )} \right] \right] 
= \mathbb{E}_g\left[ \mathbb{E}_X\left[ \exp{(\sqrt{2}c \left\langle X, g \right\rangle )} \right] \right]. \]
When we condition on $g$ (treating $g$ as a fixed vector), the subgaussian norm of $\left\langle X, g 
\right\rangle$ is at most 1 by assumption ($K = 1$), so \cref{prop:2.6.6} (iv) gives 
\[ \mathbb{E}_X\left[ \exp{(\sqrt{2}c \left\langle X, g \right\rangle )} \right] 
\leq \exp{(\lVert g \rVert_{2}^2 / 4)} \]
for some absolute constant $c > 0$. Substitute this into the bound above, then our claim for the Gaussian 
replacement is proved. After that, substitute into the first equation, and the proof is complete.
\end{proof}

The Gaussian replacement trick will also be useful when we are proving concentration regarding a chaos - namely, 
the Hanson-Wright inequality:
\begin{theorem}[Hanson-Wright inequality]
\label{thm:6.2.2}
Let $A$ be an $n \times n$ matrix. Let $X = (X_1, \dots, X_n) \in \mathbb{R}^n$ be a random vector with 
independent, mean-zero, subgaussian coordinates. Then, for every $t \geq 0$, we have
\[ P(|X^TAX - \mathbb{E}\left[ X^TAX \right]| \geq t) \leq 
2 \exp{\left[ -c \min_{}\left( \frac{t^2}{K^4 \lVert A \rVert_{F}^2}, \frac{t}{K^2 \lVert A \rVert_{}} 
\right) \right]}, \]
where $K = \max_{i} \lVert X_i \rVert_{\psi_2}$.
\end{theorem}

The proof will be based on bounding the MGF of $X^TAX$. Here is the plan:
\begin{enumerate}
	\item replace $X^TAX$ by $X^TAX'$ by decoupling;
	\item replace $X^TAX'$ by $g^T Ag'$ using Gaussian replacement, for $g \sim N(0, I_n)$;
	\item compute $g^TAg'$ by diagonalizing $A$ using the rotational invariance of $N(0, I_n)$.
\end{enumerate}

We start with part (b):
\begin{lemma}[Gaussian replacement]
\label{lem:6.2.3}
Let $A$ be an $n \times n$ matrix. Let $X$ be a mean zero, subgaussian random vector in $\mathbb{R}^n$ with 
$\lVert X \rVert_{\psi_2} \leq K$, and $X'$ be its independent copy. Let $g, g' \sim N(0, I_n)$ be independent. 
Then for any $\lambda \in \mathbb{R}$, 
\[ \mathbb{E}\left[ \exp{(\lambda X^TAX')} \right] \leq \mathbb{E}\left[ \exp{(CK^2 \lambda g^TAg')} \right]. \]
\end{lemma}

\begin{proof}
Condition on $X'$ and take expectation over $X$, which we denote $\mathbb{E}_X$. Then the random variable 
$\left\langle X, AX' \right\rangle$ is (conditionally) subgaussian, with subgaussian norm $\leq K 
\lVert AX' \rVert_{2}$. Then \cref{prop:2.6.6} (iv) gives 
\[ \mathbb{E}_X\left[ \exp{(\lambda X^TAX')} \right] \leq \exp{(C \lambda^2 K^2 \lVert AX' \rVert_{2}^2)}, 
\ \lambda \in \mathbb{R}. \]
Compare the above to the normal MGF formula. Applied to the normal random variable $g^TAX' = \left\langle 
g, AX' \right\rangle$ (still conditionally on $X'$), it gives 
\[ \mathbb{E}_g\left[ \exp{(\mu g^T AX')} \right] = \exp{(\mu^2 \lVert AX' \rVert_{2}^2 / 2)}, \ \mu 
\in \mathbb{R}. \]
Setting $\mu = \sqrt{2C}K \lambda$, we match the right hand sides of the two equations above and obtain 
\[ \mathbb{E}_X\left[ \exp{(\lambda X^TAX')} \right] = \mathbb{E}_g\left[ \exp{(\sqrt{2C}K \lambda g^TAX')} 
\right]. \]
Then, taking expectation over $X'$ on both sides, we see that we have replace $X$ by $g$ in the chaos, at a cost 
of the factor $\sqrt{2C}K$. Repeating the same thing for $X'$, we can replace $X'$ with $g'$ and get another 
factor of $\sqrt{2C}K$.
\end{proof}

We now move to step (c):
\begin{lemma}[MGF of a Gaussian quadratic form]
\label{lem:6.2.4} Let $A = (a_{ij})$ be an $n \times n$ matrix, and let $g, g' \sim N(0, I_n)$ be independent. 
Then 
\[ \mathbb{E}\left[ \exp{(\lambda g^TAg')} \right] \leq \exp{(\lambda^2 \lVert A \rVert_{F}^2)} 
\text{ whenever } |\lambda| \leq \frac{1}{2 \lVert A \rVert_{}}. \]
\end{lemma}

\begin{proof}
Let's use rotational invariance of the normal distribution do diagonalize $A$. With its singular value 
decomposition, $A = U \Sigma V^T$, we can write 
\[ g^T Ag' = (U^T g)^T \Sigma (V^T g'). \]
By the rotational invariance of the normal distribution (\cref{prop:3.3.1}), $U^T g$ and $V^T g'$ are 
independent standard normal random vectors in $\mathbb{R}^n$. So, 
\[ g^T Ag \sim g^T \Sigma g' = \sum_{i = 1}^{n} \sigma_i g_ig_i^T. \]
This is a sum of independent random variables, so 
\[ \mathbb{E}\left[ \exp{(\lambda g^TAg')} \right] 
= \mathbb{E}\left[ \prod_{i}^{} \mathbb{E}\left[ \exp{(\lambda \sigma_i g_ig_i^T)} \right] \right] 
= \prod_{i}^{} \mathbb{E}\left[ \exp{(\lambda \sigma_i g_ig_i^T)} \right]. \]
Now, for each $i$ and $t \in \mathbb{R}$, we have 
\[ \mathbb{E}\left[ \exp{(tg_ig_i')} \right] = \mathbb{E}\left[ \exp{\left( \frac{t^2 g_i^2}{2} \right)}
\right] = \frac{1}{\sqrt{1 - t^2}} \leq \exp{(t^2)} \text{ if } t^2 \leq \frac{1}{2}. \]
The first identity is done by conditioning on $g_i$ and using the MGF formula for the normal random variable 
$g'$; the other steps are just direct calculations. Substituting this bound with $t = \lambda \sigma_i$ into 
the product, we get 
\[ \mathbb{E}\left[ \exp{(\lambda g^T Ag')} \right] \leq \exp{\left( \lambda^2 \sum_{i}^{} \sigma_i^2 \right)} 
\text{ if } \lambda^2 \leq \frac{1}{2 \max_{} \sigma_i^2}. \]
Since $\sigma_i$ are the singular values of $A$, $\sum_{i}^{} \sigma_i^2 = \lVert A \rVert_{F}^2$ and 
$\max_{i} \sigma_i = \lVert A \rVert_{}$, hence the lemma is proved.
\end{proof}

Now we move to the main proof!
\begin{proof}[Proof of Hanson-Wright inequality]
Without loss of generality, assume $K = 1$. As usual, it is enough to bound the one-sided tail 
\[ p := P(X^TX - \mathbb{E}\left[ X^TAX \right] \geq t). \]
This is because we can find the lower tail by just replacing $A$ with $-A$. By combining the two tails, the 
proof would be complete.

In terms of the entries of $A = (a_{ij})$, we have 
\[ X^TAX = \sum_{i, j}^{}a_{ij}X_iX_j \text{ and } \mathbb{E}\left[ X^TAX \right] = \sum_{i}^{} a_{ii} 
\mathbb{E}\left[ X_i^2 \right], \]
where we used the mean zero assumption and independence. So 
\[ X^TAX - \mathbb{E}\left[ X^TAX \right] = \sum_{i}^{} a_{ii}(X_i^2 - \mathbb{E}\left[ X_i^2 \right]) 
+ \sum_{i \neq j}^{} a_{ij} X_iX_j. \]
The problem then reduces to estimating the diagonal and off-diagonal sums:
\[ p \leq P \left( \sum_{i}^{} a_{ii}(X_i^2 - \mathbb{E}\left[ X_i^2 \right]) \geq t/2 \right) 
+ P \left( \sum_{i \neq j}^{} a_{ij} X_iX_j \geq t/2 \right) := p_1 + p_2. \]

Let's bound these probabilities!

\textbf{Step 1: Diagonal sum.} Since $X_i$ are independent and subgaussian, $X_i^2 - \mathbb{E}\left[ X_i^2 
\right]$ are independent, mean-zero, and subexponential. Also, 
\[ \lVert X_i^2 - \mathbb{E}\left[ X_i^2 \right] \rVert_{\psi_2} \lesssim \lVert X_i^2 \rVert_{\psi_1} 
\lesssim \lVert X_i \rVert_{\psi_2}^2 \lesssim 1. \]
(The above follows from centering andcref{lem:2.8.5}). Then, Bernstein's inequality (\cref{cor:2.9.2}) gives 
\[ p_1 \leq \exp{\left[ -c \min_{}\left( \frac{t^2}{\sum_{i}^{}a_{ii}^2}, \frac{t}{\max_{i}|a_{ii}|} \right) 
\right]} \leq \exp{\left[ -c \min_{}\left( \frac{t^2}{\lVert A \rVert_{F}^2}, \frac{t}{\lVert A \rVert_{}} 
\right) \right]}. \]

\textbf{Step 2: Off-diagonal sum.} Now we bound the off-diagonal sum
\[ S := \sum_{i \neq j}^{}a_{ij}X_iX_j. \]
Let $\lambda > 0$ be a parameter to be determined later. By Merkov's inequality, we have 
\[ p_2 = P(S \geq t/2) = p(\lambda S \geq \lambda t/2) \leq \exp{(-\lambda t/2)}\mathbb{E}\left[ 
\exp{(\lambda S)} \right]. \]
We get 
\begin{align*}
	\mathbb{E}\left[ \exp{(\lambda S)} \right] 
	&\leq \mathbb{E}\left[ \exp{(4 \lambda X^TAX')} \right] \quad (\text{By decoupling}) \\
	&\leq \mathbb{E}\left[ \exp{(C_1 \lambda g^TAg')} \right] \quad (\text{By \cref{lem:6.2.3}}) \\
	&\leq \exp{(C \lambda^2 \lVert A \rVert_{F}^2)} \quad (\text{By \cref{lem:6.2.4}})
\end{align*}
whenever $|\lambda| \leq \frac{1}{2 \lVert A \rVert_{}}$. Putting this bound into the exponential bound above, 
we obtain 
\[ p_2 \leq \exp{(-\lambda t/2 + C \lambda^2 \lVert A \rVert_{F}^2)}. \]
Optimizing over $0 \leq \lambda \leq \frac{1}{2 \lVert A \rVert_{}}$, we conclude that 
\[ p_2 \leq \exp{\left[ -c \min_{}\left( \frac{t^2}{\lVert A \rVert_{F}^2}, \frac{t}{\lVert A \rVert_{}}
\right) \right]}. \]
To summarize, we obtained the desired bounds for the probabilities of the diagonal deviation $p_1$ and the 
off-diagonal deviation $p_2$. Putting them together, we complete the proof of \cref{thm:6.2.2}.
\end{proof}



% ----------6.3----------
\subsection{Symmetrization}
A random variable $X$ is called \underline{symmetric} if it has the same distribution as $-X$. A basic example 
is the Rademacher random variable, which takes values $-1$ and $1$ with equal probabilities. Mean-zero normal 
random variables are also symmetric, while the exponential and Poisson distributions are not.

This section introduces \underline{symmetrization}, a useful trick for reducing problems to symmetric 
distributions - and sometimes even to the Rademacher distribution. It is based on the following:

\begin{lemma}[Constructing symmetric distributions]
\label{lem:6.3.1}
Let $X$ be a random variable and $\xi$ be an independent Rademacher random variables. Then 
\begin{enumerate}
	\item $\xi X$ and $\xi |X|$ are identically distributed and symmetric.
	\item If $X$ is symmetric, both $\xi X$ and $\xi |X|$ have the same distribution as $X$. 
	\item If $X'$ is an independent copy of $X$, then $X - X'$ is symmetric.
\end{enumerate}
\end{lemma}

\begin{proof}
We'll check that $\xi X$ is symmetric. For any interval $A \subset \mathbb{R}$, the law of total probability 
gives 
\begin{align*}
	P(\xi X \in A) 
	&= P(\xi X \in A | \ \xi = 1) \cdot \frac{1}{2} + P(\xi X \in A | \ \xi = -1) \cdot \frac{1}{2} \\
	&= \frac{1}{2}(P(X \in A) + P(-X \in A)).
\end{align*}
Let's also do this for $-\xi X$: 
\begin{align*}
	P(-\xi X \in A) 
	&= P(-\xi X \in A | \ \xi = 1) \cdot \frac{1}{2} + P(-\xi X \in A | \ \xi = -1) \cdot \frac{1}{2} \\
	&= \frac{1}{2}(P(-X \in A) + P(X \in a)).
\end{align*}
Therefore $\xi X$ and $-\xi X$ have the same CDF, meaning they have the same distribution.

The rest of the proof is in Exercise 6.16.
\end{proof}

\begin{lemma}[Symmetrization]
\label{lem:6.3.2}
Let $X_1, \dots, X_N$ be independent, mean zero random vectors in a normed space, and let $\varepsilon_1, 
\dots, \varepsilon_N$ be independent Rademacher random variables. Then 
\[ \frac{1}{2}\mathbb{E}\left[ \left\lVert \sum_{i = 1}^{N} \varepsilon_i X_i \right\rVert_{} \right] 
\leq \mathbb{E}\left[ \left\lVert \sum_{i = 1}^{N} X_i \right\rVert_{} \right] 
\leq 2 \mathbb{E}\left[ \left\lVert \sum_{i = 1}^{N} \varepsilon_i X_i \right\rVert_{} \right]. \]
\end{lemma}

\begin{proof}
(\textbf{Upper bound}) Let $(X_i')$ be an independent copy of $(X_i)$. Since $\sum_{i = 1}^{} X_i'$ has mean 
zero, we have 
\[ p := \mathbb{E}\left[ \left\lVert \sum_{i}^{} X_i \right\rVert \right] 
\leq \mathbb{E}\left[ \left\lVert \sum_{i}^{} X_i - \sum_{i}^{} X_i' \right\rVert \right] 
= \mathbb{E}\left[ \left\lVert \sum_{i}^{} (X_i - X_i') \right\rVert \right]. \]
The inequality above comes from the fact that for independent random vectors $Y$ and $Z$, 
\[ \mathbb{E}[Z] = 0 \implies \mathbb{E}[\lVert Y \rVert_{}] \leq \mathbb{E}[\lVert Y + Z \rVert_{}]. \]
Since $X_i - X_i'$ are symmetric random vectors, they have the same distribution as 
$\varepsilon_i (X_i - X_i')$ by \cref{lem:6.3.1} (b). Then 
\begin{align*}
	p 
	&\leq \mathbb{E}\left[ \left\lVert \sum_{i}^{} \varepsilon_i (X_i - X_i') \right\rVert \right] \\
	&\leq \mathbb{E}\left[ \left\lVert \sum_{i}^{} \varepsilon_i X_i \right\rVert \right] 
	+ \mathbb{E}\left[ \left\lVert \sum_{i}^{} \varepsilon_i X_i' \right\rVert \right] \quad 
	\text{(Triangle inequality)} \\
	&= 2 \mathbb{E}\left[ \left\lVert \sum_{i}^{} \varepsilon_i X_i \right\rVert \right] \quad 
	\text{(The two terms are identically distributed)}.
\end{align*}

(\textbf{Lower bound}) The argument is similar as the proof for the upper bound:
\begin{align*}
	\mathbb{E}\left[ \left\lVert \sum_{i}^{} \varepsilon_i X_i \right\rVert \right] 
	&\leq \mathbb{E}\left[ \left\lVert \sum_{i}^{} \varepsilon_i (X_i - X_i') \right\rVert \right] \\
	&= \mathbb{E}\left[ \left\lVert \sum_{i}^{} (X_i - X_i') \right\rVert \right] \quad 
	\text{(Same distribution)} \\
	&\leq \mathbb{E}\left[ \left\lVert \sum_{i}^{} X_i \right\rVert \right] 
	+ \mathbb{E}\left[ \left\lVert \sum_{i}^{} X_i' \right\rVert \right] \quad \text{(Triangle inequality)} \\
	&= 2 \mathbb{E}\left[ \left\lVert \sum_{i}^{} X_i \right\rVert \right] \quad 
	\text{(Identical distribution)}.
\end{align*}
Question: Where did we use $X_i$'s independence? Do we need mean zero for both upper and lower bounds?
\end{proof}

There are also other versions of symmetrization lemmas (Exercises 6.19-6.21).



% ----------6.4----------
\subsection{Random Matrices with non-i.i.d. Entries}
A typical application of symmetrization consist of two steps: First, replace random variables $X_i$ with 
symmetric ones $\varepsilon_i X_i$, then condition on $X_i$ so that all randomness comes from the signs 
$\varepsilon_i$. Hence this reduces the problems to Rademacher random variables. To illustrate this techinique, 
let's bound the operator norm of a random matric with independent, non-identically distributed entries:

\begin{theorem}[Norm of random matrices with non-i.i.d. entries]
\label{thm:6.4.1}
Let $A$ be an $n \times n$ symmetric random matrix with independent, mean zero entries above and on the 
diagonal. Then 
\[ \mathbb{E}\left[ \max_{i} \lVert A_i \rVert_{2} \right] \leq \mathbb{E}\left[ \lVert A \rVert_{} \right] 
\leq C \sqrt{\log_{}{n}} \cdot \mathbb{E}\left[ \max_{i} \lVert A_i \rVert_{2} \right], \]
where $A_i$ denotes the rows of $A_i$.
\end{theorem}

\begin{proof}
The lower bound is already done in Exercise 4.7.

For the upper bound, we will use symmetrization and the matrix Khintchine inequality (\cref{thm:5.4.14}). 

Let's decompose $A$ entry-by-entry, keeping symmetry in mind, like the proof of \cref{thm:5.5.1}. Denote the 
standard basis of $\mathbb{R}^n$ by $e_1, \dots, e_n$, then $A$ can be expressed as a sum of independent, 
mean zero random matrices:
\[ A = \sum_{i \leq j}^{} Z_{ij}, \text{ where } Z_{ij} = \begin{cases}
	A_{ij}(e_i e_j^T + e_j e_i^T) &\text{ if } i < j, \\
	A_{ii}e_i e_i^T &\text{ if } i = j
\end{cases}. \]
By applying symmetrization (\cref{lem:6.3.2}), we get 
\[ \mathbb{E}\left[ \lVert A \rVert_{} \right] 
= \mathbb{E}\left[ \left\lVert \sum_{i \leq j}^{} Z_{ij} \right\rVert \right] 
\leq 2 \mathbb{E}\left[ \left\lVert \sum_{i \leq j}^{} \varepsilon_{ij} Z_{ij} \right\rVert \right] 
\quad (*) \]
where $\varepsilon_{ij}$ are independent Rademacher random variables.

Condition on $(Z_{ij})$, apply the matrix Khintchine inequality (\cref{thm:5.4.14}) for $p = 1$, and take 
expectation over $(Z_{ij})$ using the law of total expectation, which gives 
\[ \mathbb{E}\left[ \left\lVert \sum_{i \leq j}^{} \varepsilon_{ij} Z_{ij} \right\rVert \right] 
\leq C \sqrt{\log_{}{n}}\mathbb{E}\left[ \left\lVert \sum_{i \leq j}^{}Z_{ij}^2 \right\rVert^{1/2} \right]. 
\quad (**) \]
Since $(Z_{ij})$ is a diagonal matrix, 
\[ Z_{ij}^2 = \begin{cases}
	A_{ij}^2 (e_i e_j^T + e_je_i^T) &\text{ if } i < j, \\
	A_{ii}^2 e_i e_i^T &\text{ if } i = j
\end{cases}. \]
Therefore, 
\[ \sum_{i \leq j}^{} Z_{ij}^2 
= \sum_{i = 1}^{n} \left( \sum_{j = 1}^{n} A_{ij}^2 \right) e_ie_i^T 
= \sum_{i = 1}^{n} \lVert A_i \rVert_{2}^2 e_i e_i^T. \]
In other words, this is a diagonal matrix with diagonal entries equal to $\lVert A_i \rVert_{2}^2$. Since 
the operator norm of a diagonal matrix is the maximal absolute value of its entries, we get 
\[ \left\lVert \sum_{i \leq j}^{} Z_{ij}^2 \right\rVert = \max_{i} \lVert A_i \rVert_{2}^2. \]
Substitute the bound above into (**) then into (*) completes the proof.
\end{proof}

There is more practice on symmetrization as well (Exercises 6.22-6.29).



% ----------6.5----------
\subsection{Application: Matrix Completion}
Matrix completion is the process of recovering missing entries from a partially observed matrix. Of course, 
this is not possible without knowing something extra about the matrix. Let's show that for low-rank matrices, 
we can recover the missing entries algorithmically.

To describe the problem mathematically, consider an $n \times n$ matrix $X$ with 
\[ \mathrm{rank}(X) = r \]
where $r \ll n$. Suppose we are shown a few \textit{randomly chosen entries} of $X$. Each entry $X_{ij}$ is 
revealed to us independently with some probability $p \in (0, 1)$ and is hidden from us with probability 
$1 - p$. In other words, assume that we observe the $n \times n$ matrix $Y$ with entries 
\[ Y_{ij} = \delta_{ij}X_{ij} \text{ where } \delta_{ij} \sim_{i.i.d.} \mathrm{Ber}(p). \]
These $\delta_{ij}$ are \textit{selectors}. If 
\[ p = \frac{m}{n^2} \]
then we observe $m$ entries on average.

The question is, how can we recover $X$ from $Y$? Although $X$ has small rank $r$, $Y$ may not have small rank. 
To fix this, we can pick the best rank $r$ approximation to $Y$. Properly scaled, this gives a good estimate of 
the original matrix $X$:

\begin{theorem}[Matrix completion]
\label{thm:6.5.1} 
Let $\Hat{X}$ be a best rank $r$ approximation to $p^{-1}Y$. Then 
\[ \mathbb{E}\left[ \frac{1}{n}\lVert \Hat{X} - X \rVert_{F} \right] \leq 
C \sqrt{\frac{rn \log_{}{n}}{m}} \lVert X \rVert_{\infty}, \]
as long as $m \geq n \log_{}{n}$. Here $\lVert X \rVert_{\infty} = \max_{i, j} |X_{ij}|$ is the largest entry, 
NOT the usual matrix infinity norm!
\end{theorem}

Before we prove this, note that the recovery error 
\[ \frac{1}{n}\lVert \Hat{X} - X \rVert_{F} = \left( \frac{1}{n}\sum_{i, j = 1}^{n^2} 
\sum_{i, j = 1}^{n} |\Hat{X}_{ij} - X_{ij}|62 \right)^{1/2} \]
represents the average error per entry (in the $L^2$ sense). If we choose the average number of observed entries 
$m$ so that 
\[ M \geq C' rn \log_{}{n} \]
with large constant $C'$, then \cref{thm:6.5.1} guarantees that the average error is much smaller than 
$\lVert X \rVert_{\infty}$. So, matrix completion is possible if the number of observed entries exceeds $rn$ by 
a logarithmic margin.

\begin{proof}
We first bound the recovery error in the operator norm, and then pass to the Frobenius norm using the low-rank 
assmumption.

\textbf{Step 1: Bounding the error in the operator norm.} Using the triangle inequality, we can split the error 
as follows:
\[ \lVert \Hat{X} - X \rVert_{} \leq \lVert \Hat{X} - p^{-1}Y \rVert_{} + \lVert p^{-1}Y - X \rVert_{}. \]
Since we have chosen $\Hat{X}$ as a best rank $r$ approximation to $p^{-1}Y$, the second summand dominates, i.e. 
$\lVert \Hat{X} - p^{-1}Y \rVert_{} \leq \lVert p^{-1}Y - X \rVert_{}$, so we have 
\[ \lVert \Hat{X} - X \rVert_{} \leq 2 \lVert p^{-1}Y - X \rVert_{} = \frac{2}{p}\lVert Y - pX \rVert_{}. \]
Note that the matrix $\Hat{X}$, which is tricky to handle, is gone in the bound. Instead, we get $Y - pX$, which 
is easier to understand since its entries, 
\[ (Y - pX)_{ij} = (\delta_{ij} - p)X_{ij}, \]
are independent, mean-zero random variables. Using \cref{thm:6.4.1} (more precisely, Exercise 6.28), we get 
\[ \mathbb{E}\left[ \lVert Y - pX \rVert_{} \right] \leq C \sqrt{\log_{}{n}} \left( 
\mathbb{E}\left[ \max_{i = 1, \dots, n}\lVert (Y - pX)_{i:} \rVert_{2} \right] 
+ \mathbb{E}\left[ \max_{j = 1, \dots, n}\lVert (Y - pX)_{:j} \rVert_{2} \right] \right). \quad (*) \]
To bound the norms of the rows of $Y - pX$, we write them as 
\[ \lVert (Y - pX)_{i:} \rVert_{2}^2 = \sum_{j = 1}^{n}(\delta_{ij} - p)^2 X_{ij}^2 
\leq \sum_{j = 1}^{n} (\delta_{ij} - p)^2 \cdot \lVert X \rVert_{\infty}^2, \]
and similarly for columns. These sums of independent random variables can be easily bounded using Bernstein's 
(or Chernoff's) inequality, which yields (Exercise 6.30)
\[ \mathbb{E}\left[ \max_{i = 1, \dots, n}\sum_{j = 1}^{n}(\delta_{ij} - p)^2 \right] \lesssim pn. \]
Combining with a similar bound for the columns and substituting into $(*)$, we obtain 
\[ \mathbb{E}\left[ \lVert Y - pX \rVert_{} \right] \lesssim \sqrt{pn \log_{}{n}}\lVert X \rVert_{\infty}. \]
Then, by the bound for $\lVert \Hat{X} - X \rVert_{}$ from earlier, we get 
\[ \mathbb{E}\left[ \lVert \Hat{X} - X \rVert_{} \right] \lesssim \sqrt{\frac{n \log_{}{n}}{p}} \lVert 
X \rVert_{\infty}. \]

\textbf{Passing to the Frobenius norm.} We have not used the low rank assumption yet, so we'll do this now. 
Since $\mathrm{rank}(X) \leq r$ by assumption and $\mathrm{rank}(\Hat{X}) \leq r$ by construction, we have 
(Exercise 4.4)
\[ \mathrm{rank}(\Hat{X} - X) \leq 2r \implies \lVert \Hat{X} - X \rVert_{F} 
\leq \sqrt{2r} \lVert \Hat{X} - X \rVert_{}. \]
Taking expectations and using the bound on the error in the operator norm from step 1, we get 
\[ \mathbb{E}\left[ \lVert \Hat{X} - X \rVert_{F} \right] \lesssim \sqrt{\frac{rn \log_{}{n}}{p}} 
\lVert X \rVert_{\infty}. \]
Dividing both sides by $n$, we can rewrite this bound as 
\[ \mathbb{E}\left[ \frac{1}{n}\lVert \Hat{X} - X \rVert_{F} \right] \lesssim 
\sqrt{\frac{rn \log_{}{n}}{pn^2}} \lVert X \rVert_{\infty}. \]
From the definition above, $pn^2 = m$ so plugging in finishes the proof.
\end{proof}

\begin{remark}[Extensions]
\label{rmk:6.5.2}
\cref{thm:6.5.1} can be extended and improved in many ways, such as to rectangular matrices (Exercise 6.31) and 
matrices with noisy observations (Exercise 6.32). It is less trivial but possible to remove the logarithmic 
factor from the error bound, and acheive zero eror for noiseless observations!
\end{remark}



% ----------6.6----------
\subsection{Contraction Principle}
There is one more useful inequality the text covers in the chapter:

\begin{theorem}[Contraction principle]
\label{thm:6.6.1}
Let $x_1, \dots, x_N$ be any vectors in a normed space, $(a_1, \dots, a_N) \in \mathbb{R}^N$, and 
$\varepsilon_1, \dots, \varepsilon_N$ be independent Rademacher random variables. Then 
\[ \mathbb{E}\left[ \left\lVert \sum_{i = 1}^{N} a_i \varepsilon_i x_i \right\rVert \right] 
\leq \lVert a \rVert_{\infty} \cdot \mathbb{E}\left[ \left\lVert 
\sum_{i = 1}^{N} \varepsilon_i x_i \right\rVert \right]. \]
\end{theorem}

\begin{proof}
WLOG, assume that $\lVert a \rVert_{\infty} \leq 1$. Define the function 
\[ f(a) := \mathbb{E}\left[ \left\lVert \sum_{i = 1}^{N} a_i \varepsilon_i x_i \right\rVert \right]. \]
Then $f: \mathbb{R}^n \to \mathbb{R}$ is convex (Exercise 6.35).

We want to bound for $f$ the set of points $a$ satisfying $\lVert a \rVert_{\infty} \leq 1$, i.e. on the unit 
cube $[-1, 1]^N$. By the maximum principle (Exercises 1.4 \& 1.5), the maximum of a convex function on the 
cube is attained at a vertex, where all $a_i = \pm 1$. For such $a$, the random variables $(\varepsilon_i a_i)$ 
have the same distribution as $\varepsilon_i$ by symmetry. Thus 
\[ \mathbb{E}\left[ \left\lVert \sum_{i = 1}^{N} a_i \varepsilon_i x_i \right\rVert \right] 
= \mathbb{E}\left[ \left\lVert \sum_{i = 1}^{N} \varepsilon_i x_i \right\rVert \right], \]
thus 
\[ f(a) \leq \mathbb{E}\left[ \left\lVert \sum_{i = 1}^{N} \varepsilon_i x_i \right\rVert \right] 
\text{ whenever } \lVert a \rVert_{\infty} \leq 1, \]
which completes the proof.
\end{proof}

As an application, we can prove a version of symmetrization but with Gaussian random variables 
$g_i \sim N(0, 1)$ instead of Rademachers.

\begin{lemma}[Symmetrization with Gaussians]
\label{lem:6.6.2}
Let $X_1, \dots, X_N$ be independent, mean zero random vectors in a normed space. Let $g_1, \dots, g_N 
\sim N(0, 1)$ be independent Gaussian random variables, which are also independent of $X_i$. Then 
\[ \frac{c}{\sqrt{\log_{}{N}}} \mathbb{E}\left[ \left\lVert \sum_{i = 1}^{N} g_iX_i \right\rVert \right] 
\leq \mathbb{E}\left[ \left\lVert \sum_{i = 1}^{N} X_i \right\rVert \right] 
\leq 3 \mathbb{E}\left[ \left\lVert \sum_{i = 1}^{N} g_i X_i \right\rVert \right]. \]
\end{lemma}

\begin{proof}
\textbf{(Upper bound)} By symmetrization (\cref{lem:6.3.2}), we have 
\[ E := \mathbb{E}\left[ \left\lVert \sum_{i = 1}^{N} X_i \right\rVert \right] 
\leq 2 \mathbb{E}\left[ \left\lVert \sum_{i = 1}^{N} \varepsilon_i X_i \right\rVert \right]. \]
To interject Gaussian random variables, recall that $\mathbb{E}\left[ |g_i| \right] = \sqrt{2 / \pi}$. Then 
we can continue the bound as follows: 
\begin{align*}
	E 
	&\leq 2 \sqrt{\frac{\pi}{2}} \mathbb{E}_X\left[ \left\lVert \sum_{i = 1}^{N} \varepsilon_i 
	\mathbb{E}_g\left[ |g_i| \right] X_i \right\rVert \right] \\
	&\leq 2 \sqrt{\frac{\pi}{2}} \mathbb{E}\left[ \left\lVert \sum_{i = 1}^{N} 
	\varepsilon_i |g_i| X_i \right\rVert \right] \quad \text{(Jensen inequality)} \\
	&= 2 \sqrt{\frac{\pi}{2}} \mathbb{E}\left[ \left\lVert \sum_{i = 1}^{N} 
	g_i X_i \right\rVert \right]. 
\end{align*}
The last equality holds since the random variables $(\varepsilon_i |g_i|)$ have the same joint distribution as 
$(g_i)$ (\cref{lem:6.3.1} (b)).

\textbf{(Lower bound)} We have 
\begin{align*}
	\mathbb{E}\left[ \left\lVert \sum_{i = 1}^{N} g_i X_i \right\rVert \right] 
	&= \mathbb{E}\left[ \left\lVert \sum_{i = 1}^{N} \varepsilon_i g_i X_i \right\rVert \right] \quad 
	\text{(Symmetry of $g_i$)} \\
	&\leq \mathbb{E}_g\left[ \mathbb{E}_X\left[ \lVert g \rVert_{\infty} 
	\mathbb{E}_{\varepsilon}\left[ \left\lVert \sum_{i = 1}^{N} \varepsilon_i X_i 
	\right\rVert \right] \right] \right] \quad \text{(\cref{thm:6.6.1})} \\
	&= \mathbb{E}_g\left[ \lVert g \rVert_{\infty} \mathbb{E}_{\varepsilon}\left[ \mathbb{E}_X\left[ 
	\left\lVert \sum_{i = 1}^{N} \varepsilon_i X_i \right\rVert \right] \right] \right] \quad 
	\text{(Independence)} \\
	&\leq 2 \mathbb{E}_g\left[ \lVert g \rVert_{\infty} \mathbb{E}_X\left[ \left\lVert 
    \sum_{i = 1}^{N} X_i\right\rVert \right] \right] \quad \text{(\cref{lem:6.3.2})} \\
	&= 2 \mathbb{E}\left[ \lVert g \rVert_{\infty} \right] \cdot 
	\mathbb{E}\left[ \left\lVert \sum_{i = 1}^{N} X_i \right\rVert \right] \quad \text{(Independence).}
\end{align*}
Moreover, by \cref{prop:2.7.6}, 
\[ \mathbb{E}\left[ \lVert g \rVert_{\infty} \right] \leq C \sqrt{\log_{}{N}}. \]
Plugging back gives the result.
\end{proof}

\begin{remark}[Log factor is unavoidable]
\label{rmk:6.6.3}
The logarithmic factor in \cref{lem:6.6.2} is necessary and optimal in general (Exercise 6.37), making 
Gaussian symmetrization weaker than Rademacher's.
\end{remark}

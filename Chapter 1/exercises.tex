\documentclass{article}
\usepackage{amsmath}
\usepackage{amssymb}
\usepackage[a4paper, margin=1in]{geometry}
\usepackage{graphicx}
\usepackage{minted}
\renewcommand\thesubsection{(\alph{subsection})}
\renewcommand\thesubsubsection{(\roman{subsubsection})}
\renewcommand{\labelenumi}{(\alph{enumi})}
\renewcommand{\labelenumii}{(\roman{enumii})}
\setlength{\parindent}{0pt}

\title{Chapter 1 Exercises}
\author{Gallant Tsao}
\date{}

\begin{document}
\maketitle


% Exercise 1
\section*{Exercise 1}
Let $x_1, x_2 \in \text{conv}(T)$, and $\lambda \in [0, 1]$. Then there exists $j, k \in \mathbb{N}$ such that 
\[ x_1 = a_1 y_1 + \cdots + a_j y_j, a_i \geq 0, \sum_{i = 1}^{j} a_i = 1, \]
\[ x_2 = b_1 z_1 + \cdots + b_k z_k, b_I \geq 0, \sum_{i = 1}^{k} b_i = 1. \]
Then we get 
\[ \lambda x_1 + (1 - \lambda) x_2 
= \lambda \sum_{i = 1}^{j} a_i y_i + (1 - \lambda)\sum_{i = 1}^{k} b_i z_i. \]
From the formulation above, 
\[ a_i \geq 0 \implies \lambda a_i \geq 0, \ b_i \geq 0 \implies (1 - \lambda) b_i \geq 0. \]
Moreover, when summing up the coefficients, 
\[ \lambda \sum_{i = 1}^{j} a_i + (1 - \lambda) \sum_{i = 1}^{k} b_i = \lambda + (1 - \lambda) = 1. \]
Therefore $\lambdax_1 + (1 - \lambda)x_2 \in \text{conv}(T)$. Here we assumed that without loss of generality, there are no shared 
points between $x_1$ and $x_2$. If there were to be shared points, it would not have affected our analysis because each coefficient 
in the convex combination will still be greater than 0, and also their sum will be 1.


% Exercise 2
\newpage
\section*{Exercise 2}
Let $f_1, \dots, f_m$ be convex functions, and $g: K \to \mathbb{R}$ be defined as 
\[ g(x) = \max_{x}(f_1(x), \cdots, f_m(x)). \]
Let $x, y \in K$, and let $\lambda \in [0, 1]$. Then 
\begin{align*}
	g(\lambda x + (1 - \lambda)y) 
	&= \max_{}(f_1(\lambda x + (1 - \lambda)y), \cdots, f_m(\lambda x + (1 - \lambda)y)) \\
	&\leq \max_{}(\lambda f_1(x) + (1 - \lambda)f_1(y), \cdots, \lambda f_m(x) + (1 - \lambda) f_m(y)) \\
	&\leq \max_{}(\lambda f_1(x), \cdots, \lambda f_m(x)) + \max_{}((1 - \lambda)f_1(y), \cdots, (1 - \lambda)f_m(y)) \\
	&= \lambda \max_{}(f_1(x), \cdots, f_m(x)) + (1 - \lambda) \max_{}(f_1(y), \cdots, f_m(y)) \\
	&= \lambda g(x) + 1 - \lambda) g(y).
\end{align*}
Therefore $g$ is a convex function.


% Exercise 3
\newpage
\section*{Exercise 3}
\subsection*{(a)}
($\implies$) Suppose that $f$ is convex. For the base case, when $m = 2$, by the definition of convexity, the statement is true. 
For the inductive hypothesis, assume that for some $m \in \mathbb{N}$, 
\[ f \biggl( \sum_{i = 1}^{m} \lambda_i x_i \biggr) \leq \sum_{i = 1}^{m} \lambda_i f(x_i), \lambda_1 \geq 0, \sum_{i = 1}^{m} \lambda_i = 1. \]
With $\lambda_j \geq 0, \sum_{j = 0}^{m + 1} \lambda_j = 1$, without loss of generality assume that $\lambda_{m + 1} < 1$ (if not we can switch 
to another $\lambda$ that satisfies this condition). 
\begin{align*}
	f \biggl( \sum_{i = 1}^{m + 1} \lambda_i x_i \biggr) 
	&= f \biggl( (1 - \lambda_{m + 1}) \sum_{j = 1}^{m} \frac{\lambda_j}{1 - \lambda_{j + 1}} x_j + \lambda_{m + 1} x_{m + 1} \biggr) \\
	&\leq (1 - \lambda_{m + 1})  f \biggl( (1 - \lambda_{m + 1}) \sum_{j = 1}^{m} \frac{\lambda_j}{1 - \lambda_{j + 1}} x_j \biggr) 
	+ \lambda_{m + 1} f(x_{m + 1}) \quad \text{(Base case)} \\
	&\leq (1 - \lambda_{m + 1}) \sum_{j = 1}^{m} \frac{\lambda_j}{1 - \lambda_{m + 1}}f(x_j) + \lambda_{m + 1}f(x_{m + 1}) \quad 
	\text{(Inductive step)} \\
	&= \sum_{j = 1}^{m} \lambda_j f(x_j) + \lambda_{m + 1} f(x_{m + 1}) \\
	&= \sum_{j = 1}^{m + 1} \lambda_j f(x_j).
\end{align*}
($\impliedby$) Take $m = 2$ and we are done.

\subsection*{(b)}
By the definition given for $X$,, let 
\[ P(X = x_i) = p_i, i = 1, \dots, n, \ p_i \geq 0, \sum_{i = 1}^{n} p_i = 1. \]
We can directly see from our construction that 
\[ \mathbb{E}[X] = \sum_{i = 1}^{n} p_i x_i. \]
Then from part (a), 
\[ f(\mathbb{E}[X]) = f \biggl( \sum_{i = 1}^{n} p_i x_i \biggr) 
\leq \sum_{i = 1}^{n} p_i f(x_i) = \mathbb{E}[f(X)]. \]


% Exercise 4
\newpage
\section*{Exercise 4}
Let $x \in \text{conv}(T)$. Then for some $m \in \mathbb{N}$, 
\[ x = \lambda_1 z_1 + \cdots + \lambda_m z_m, \ \lambda_i \geq 0, \sum_{i = 1}^{m} \lambda_i = 1. \]
Then by Jensen's Inequality from Exercise 3, 
\[ f(x) = f \biggl( \sum_{i = 1}^{m} \lambda_i z_i \biggr) \leq \sum_{i = 1}^{m} \lambda_i f(z_i) \leq \sup_{i} f(z_i). \]
Therefore we get 
\[ \sup_{x \in \text{conv}(T)} f(x) \leq \sup_{x \in T} f(x). \]
The other side ("$\geq$") is obvious because $T \subseteq \text{conv}(T)$. Therefore we get the equality.


\end{document}

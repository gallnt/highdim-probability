\documentclass{article}
\usepackage{amsmath}
\usepackage{amssymb}
\usepackage[a4paper, margin=1in]{geometry}
\usepackage{graphicx}
\usepackage{minted}
\usepackage{amsthm}
\usepackage[english]{babel}

\renewcommand\thesubsection{(\alph{subsection})}
\renewcommand\thesubsubsection{(\roman{subsubsection})}
\renewcommand{\labelenumi}{(\alph{enumi})}
\renewcommand{\labelenumii}{(\roman{enumii})}
\setlength{\parindent}{0pt}

\newtheorem*{theorem}{Theorem}
\newtheorem*{corollary}{Corollary}
\newtheorem*{lemma}{Lemma}
\newtheorem*{remark}{Remark}

\theoremstyle{definition}
\newtheorem*{definition}{Definition}

\title{Chapter 1: A Quick Refresher on Analysis \& Probability}
\author{Gallant Tsao}

\begin{document}
\maketitle
\section*{Convex Sets and Functions}

\begin{definition}[]
A subset $K \subseteq \mathbb{R}^n$ is a \underline{convex set} if, for any pair of points in $K$, the line 
segment connecting these two points is also contained in $K$, i.e. 
\[ \lambda x + (1 - \lambda) y \in K \quad \forall x, y \in K, \lambda \in [0, 1]. \]
Let $K \in \mathbb{R}^n$ be a convex subset. A function $f: K \to \mathbb{R}$ is a \underline{convex 
function} if 
\[ f(\lambda x + (1 - \lambda) y) \leq \lambda f(x) + (1 - \lambda) f(y) \quad \forall x, y \in K, 
\lambda \in [0, 1]. \]	
$f$ is \underline{concave} if the inequality above is reversed, or equivalently, if $-f$ is convex.
\end{definition}


\section*{Norms and Inner Products}




\end{document}

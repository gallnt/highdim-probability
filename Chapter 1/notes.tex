\section{A Quick Refresher on Analysis and Probability}



% ----------1.1----------
\subsection{Convex Sets and Functions}

A subset $K \subseteq \mathbb{R}^n$ is a \underline{convex set} if, for any pair of points in $K$, the line 
segment connecting these two points is also contained in $K$, i.e. 
\[ \lambda x + (1 - \lambda) y \in K \quad \forall x, y \in K, \lambda \in [0, 1]. \]
Let $K \in \mathbb{R}^n$ be a convex subset. A function $f: K \to \mathbb{R}$ is a \underline{convex 
function} if 
\[ f(\lambda x + (1 - \lambda) y) \leq \lambda f(x) + (1 - \lambda) f(y) \quad \forall x, y \in K, 
\lambda \in [0, 1]. \]	
$f$ is \underline{concave} if the inequality above is reversed, or equivalently, if $-f$ is convex.



% ----------1.2----------
\subsection{Norms and Inner Products}

The \underline{Euclidean norm} of a vector $x \in \mathbb{R}^n$ is 
\[ \|x\|_2 = \biggl( \sum_{i = 1}^{n} x_i^2 \biggr)^{1/2}. \]

The \underline{inner product (dot product)} of two vectors $x, y \in \mathbb{R}^n$ is 
\[ \langle x, y \rangle = x^T y. \]

For $p \in [1, \infty]$, the \underline{$\ell^p$ norm} of a vector $x \in \mathbb{R}^n$ is 
\[ \|x\|_p = \biggl( \sum_{i = 1}^{n} |x_i|^p \biggr)^{1/p} \quad \text{ for } p \in [1, \infty), 
\ \|x\|_{\infty} = \max_{i = 1, \dots, n} |x_i|. \]

For any vector $x, y \in \mathbb{R}^n$, 
\[ \|x + y\|_p \leq \|x\|_p + \|y\|_p. \]
It follows that the $\ell^p$ norm defines a norm on $\mathbb{R}^n$  for every $p \in [1, \infty]$.

For all vectors $x, y \in \mathbb{R}^n$, 
\[ |\langle x, y \rangle| \leq \lVert x \rVert_{2} \lVert y \rVert_{2}. \]

For all vectors $x, y \in \mathbb{R}^n$, 
\[ |\langle x, y \rangle| \leq \lVert x \rVert_{p} \lVert y \rVert_{p'} \text{ if } 
\frac{1}{p} + \frac{1}{p'} = 1 \]
where $p, p'$ are called \underline{conjugate exponents}.



% ----------1.3----------
\subsection{Random Variables and Random Vectors}

We'll do a brief review of some important concepts about random variables first:

The \underline{expectation (mean)} of a random variable $X$ is 
\[ \mathbb{E}[X] = \sum_{k = -\infty}^{\infty} k p_X(k) = \int_{-\infty}^{\infty} x f_X(x) \ dx. \]
Its \underline{variance} is 
\[ \mathrm{Var}(X) = \mathbb{E}[(X - \mathbb{E}[X])^2] = \mathbb{E}[X^2] - (\mathbb{E}[X])^2. \]

The expectation is linear: 
\[ \mathbb{E}[a_1 X_1 + \cdots + a_n X_n] = a_1 \mathbb{E}[X_1] + \cdots + a_n \mathbb{E}[X_n]. \]
For variance this is not the case. However, if the random variables are independent (or even uncorrelated): 
\[ \mathrm{Var}(a_1 X_1 + \cdots + a_n X_n) = a_1^2 \mathrm{Var}(X_1) + \cdots + a_n^2 \mathrm{Var}(X_n). \]

The simplest example of a random variable is the \textit{indicator} of a given event $E$, which is 
\[ \mathbf{1}_E(x) = \begin{cases}
	1 & \text{ if } x \in E, \\
	0 & \text{ if } x \notin E.
\end{cases} \]
Its expectation is given by 
\[ \mathbb{E}[\mathbf{1}_E] = P(E). \]

The \underline{moment generating function (mgf)} of a random variable $X$ is given by 
\[ M_X(t) = \mathbb{E}[e^{tX}], t \in \mathbb{R}. \]

For $p > 0$, the \underline{pth moment} of a random variable $X$ is $\mathbb{E}[X^p]$, and the 
\underline{pth absolute moment} is $\mathbb{E}[|X|^p]$. By taking the $p$th root of the absolute moment, 
we get the \underline{$L^p$ norm} of a random variable: 
\[ \lVert X \rVert_{L^p} = (\mathbb{E}[|X|^p])^{1/p}, \text{ and } 
\lVert X \rVert_{\infty} = \text{ess} \sup |X|, \]
where esssup denotes the essential supremum. 

The normed space consisting of all random variables on a given probability space that have finite 
$L^p$ norm is called the \underline{$L^p$ space}:
\[ L^p = \{ X: \ \lVert X \rVert_{L^p} < \infty \}. \]

The \underline{standard deviation} of a random variable $X$ is 
\[ \sigma = \sqrt{\mathrm{Var}(X)} = \lVert X - \mathbb{E}[X] \rVert_{L^2}. \]
The \underline{covariance} of two random variables $X$ and $Y$ is 
\[ \mathrm{Cov}(X, Y) = \mathbb{E}[(X - \mathbb{E}[X])(Y - \mathbb{E}[Y])] 
= \langle X - \mathbb{E}[X], Y - \mathbb{E}[Y] \rangle_{L^2}. \]

A \underline{random vector} $X = (X_1, \dots, X_n)$ is a vector whose all $n$ coordinates $X_i$ 
are random variables. Its expected value is
\[ \mathbb{E}[X] = (\mathbb{E}[X_1], \dots, \mathbb{E}[X_n]). \]
Its \underline{covariance matrix} is 
\[ \mathrm{Cov}(X) = \mathbb{E}[(X - \mathbb{E}[X]) (X - \mathbb{E}[X])^T]. \]
which is a $n \times n$ matrix whose $(i, j)$-th entry is $\mathrm{Cov}(X_i, X_j)$.


% ----------1.4----------
\subsection{Union Bound}
\begin{lemma}[Union bound]
\label{lem:1.4.1}
For any events $E_1, \dots, E_n$, we have 
\[ P \left( \bigcup_{i = 1}^n E_i \right) \leq \sum_{i = 1}^{n} P(E_i). \]
\end{lemma}

\begin{proof}
If the event $\cup_{i = 1}^n$ occurs, at least of the events $E_i$ has to occur. Therefore their respective 
indicator random variables satisfy 
\[ \mathbf{1}_{\cup_{i = 1}^n E_i} \leq \mathbf{1}_{E_i}. \]
Taking expectations and using the linearity of expectation completes the proof.
\end{proof}

\begin{example}[Dense random graphs have no isolated vertices]
\label{ex:1.4.2}
Consider the $G(n, p)$ graph from the Erdos-Renyi model, with $n \geq 2$. Show that if $p \geq 4 \ln{n}/n$ 
then there are no isolated vertices with probability at least $1 - 1/n$.
\end{example}

\begin{proof}
Call the vertices $1, \dots, n$ and let $E_i$ denote the event that vertex has no neighbors. This means that 
none of the other $n - 1$ vertices are neighbors with vertex $i$, and these $n - 1$ events are independent 
and have probability $1 - p$ each. Thus $P(E_i) = (1 - p)^{n - 1}$.

Therefore, by union bound, we have 
\begin{align*}
	P \left( \bigcup_{i = 1}^n E_i \right) 
	&\leq \sum_{i = 1}^{n} P(E_i) \\
	&= n(1 - p)^{n - 1}.
\end{align*}
\end{proof}


% ----------1.5----------
\subsection{Conditioning}

Given a probability space, the \underline{conditional probability} of an event $E$ given an event $F$ 
is 
\[ P(E|F) = \frac{P(E \cap F)}{P(F)}. \]

\begin{example}[Probability of a perfect cancellation]
\label{ex:1.5.1}
Let $a_1, \dots, a_n \in \mathbb{R}$, not all of which are zero. What is the probability that 
\[ \pm a_1 + \cdots \pm a_n = 0 \]
where the signs are chosen at random?

We can show that this probability is always bounded by $1/2$. We model the random signs as independent 
Rademacher random variables $X_1, \dots, X_n$. We claim that 
\[ P(S_n = 0) \leq \frac{1}{2} \text{ where } S_n = \sum_{i = 1}^{n} a_i X_i. \]
\end{example}

\begin{proof}
We can assume WLOG that $a_n \neq 0$ or else we can just rearrange. By conditioning on the random variables 
$X_1, \dots, X_{n - 1}$, we get that 
\[ P(S_n = 0 | \ X_1, \dots, X_{n - 1}) 
= P \left( X_n = -\frac{S_{n - 1}}{a_n} \bigg| X_1, \dots, X-{n - 1} \right) \leq \frac{1}{2}. \]
The inequality holds because $X_n$ is independent of $X_1, \dots, X_{n - 1}$, the value of $u = -S_n / a_n$ 
is fixed by conditioning, and the definition of Rademacher distribution implies that $P(X_n = u) \leq 1/2$ 
for all $u \in \mathbb{R}$. Then by applying the law of total expectation, we get 

\[ P(S_n = 0) = \mathbb{E}[P(S_n = 0 | \ X_1, \dots, X_{n - 1})] \leq \mathbb{E}[1/2] = 1/2. \]
\end{proof}
In fact, the result for \cref{ex:1.5.1} is sharp: If there are exactly two nonzero coefficients $a_i$ which 
are equal to each other, $P(S_n = 0) = 1/2$ because we need opposite signs!



% ----------1.6----------
\subsection{Probabilistic Inequalities}
Jensen inequality states for any random variable $X$ and a convex function $f: \ \mathbb{R} \to \mathbb{R}$, 
\[ f(\mathbb{E}[X]) \leq \mathbb{E}[f(X)]. \]
This also holds for any random vector taking values in $\mathbb{R}^n$ and any convex function 
$f: \mathbb{R}^n \to \mathbb{R}$.

In particular, since any norm on $\mathbb{R}^n$ is convex, we get
\[ \lVert \mathbb{E}[X] \rVert_{} \leq \mathbb{E}[\lVert X \rVert_{}]. \]

Minkowski inequality states that for any $p \in [1, \infty]$ and any random variables $X, Y \in L^p$, 
\[ \lVert X + Y \rVert_{L^p} \leq \lVert X \rVert_{L^p} + \lVert Y \rVert_{L^p}. \]

Cauchy-Schwartz inequality states that for any random variables $X, Y \in L^2$, 
\[ \lVert XY \rVert_{L^1} \leq \lVert X \rVert_{L^2} \lVert Y \rVert_{L^2}. \]

Hölder inequality generalized tha above to the $L^p$ norms. For any pair of conjugate exponents 
$p, p' \in [1, \infty]$ and any pair of random of random variables $X \in L^p$, $Y \in L^{p'}$, we have 
\[ \lVert XY \rVert_{L^1} \leq \lVert X \rVert_{L^p} \lVert Y \rVert_{L^{p'}}. \]

The \underline{cumulative distribution function (CDF)} of $X$ is 
\[ F_X(t) = P(X \leq t), t \in \mathbb{R}. \]

The following result allows us to compute expectation in terms of the tail:
\begin{lemma}[Integrated tail formula]
\label{lem:1.6.1}
Any nonnegative random variable $X$ satisfies 
\[ \mathbb{E}[X] = \int_{0}^{\infty} P(X > t) \ dt. \]
The two sides of the equation are either finite or infinite simultaneously.
\end{lemma}

\begin{proof}
We can represent any nonnegative real number $x$ via the identity 
\[ x = \int_{0}^{x} 1 \ dt = \int_{0}^{\infty} \mathbf{1}_{t < x} \ dt. \]
Replace $x$ with the random variable $X$ and taking expectations on both sides gives
\begin{align*}
	\mathbb{E}[X] 
	&= \mathbb{E}\left[ \int_{0}^{\infty} \mathbf{1}_{t < X} \ dt \right] \\
	&= \int_{0}^{\infty} \mathbb{E}[\mathbf{1}_{t < X}] \ dt \quad \text{ (Fubini-Tonelli theorem) }  \\
	&=\int_{0}^{\infty} P(t < X) \ dt.
\end{align*}
\end{proof}

There are also some other concentration inequalities: 
\begin{proposition}[Markov inequality]
For any nonnegative random variable $X$ and $t > 0$, 
\[ P(X \geq t) \leq \frac{\mathbb{E}[X]}{t}. \]
\end{proposition}

\begin{proof}
Fix $t > 0$. We can represent any real number $X$ via the identity 
\[ x = x \mathbf{1}_{x \geq t} + x \mathbf{1}_{x < t}. \]
Replacing $x$ with the random variable $X$ and taking expectation gives 
\[ \mathbb{E}[X] = \mathbb{E}[X \mathbf{1}_{X \geq t}] + \mathbb{E}[X \mathbf{1}_{X < t}] 
\geq \mathbb{E}[t \mathbf{1}_{X \geq t}] + 0 = t P(X \geq t). \]
Dividing both sides by $t$ gives the result.
\end{proof}

\begin{corollary}[Chebyshev inequality]
Let $X$ be a random variable with mean $\mu$ and variance $\sigma^2$. Then for any $t > 0$, 
\[ P(|X - \mu| \geq t) \leq \frac{\sigma^2}{t^2}. \]
\end{corollary}

\begin{proof}
By Markov inequality (\cref{lem:1.6.1}), 
\[ P((X - \mu)^2 \geq t^2) \leq \frac{\mathbb{E}[(X - \mu)^2]}{t^2} = \frac{\sigma^2}{t^2}. \]
\end{proof}

% ----------1.7----------
\subsection{Limit Theorems}

\begin{theorem}[Strong law of large numbers]
\label{thm:1.7.1}
Let $X_1, X_2, \dots$ be a sequence of i.i.d. random variables with mean $\mu$. Let $S_N = X_1 + \cdots 
+ X_N$. Then as $N \to \infty$, 
\[ \frac{S_N}{N} \to \mu \text{ almost surely. } \]
\end{theorem}

\begin{definition}[]
\label{def:1.7.2}
A random variable $X$ is a \underline{standard normal} random variable, denoted $X \sim N(0, 1)$, if 
its density is 
\[ f_X(x) = \frac{1}{\sqrt{2 \pi}}e^{-x^2 / 2}, x \in \mathbb{R}. \]
$X$ has mean zero and variance 1.

More generally, $X$ as a \underline{normal distribution} with mean $\mu$ and variance $\sigma^2$ if 
its density is 
\[ f_X(x) = \frac{1}{\sigma \sqrt{2 \pi}} e^{-\frac{(x - \mu)^2}{2 \sigma^2}}, x \in \mathbb{R}. \]
\end{definition}

\begin{theorem}[Lindeberg–Lévy CLT]
\label{thm:1.7.3}
Let $X_1, X_2, \dots$ be a sequence of i.i.d. random variables with mean $\mu$ and variance $\sigma^2$. 
Consider the sum $S_N = X_1 + \cdots + X_N$. Normalize this sum so that it has zero mean and unit variance: 
\[ Z_N := \frac{S_N - \mathbb{E}[S_N]}{\sqrt{\mathrm{Var}(S_N)}} = 
\frac{1}{\sigma \sqrt{N}} \sum_{i = 1}^{N} (X_i - \mu). \]
Then as $N \to \infty$, 
\[ Z_N \to N(0, 1) \text{ in distribution, } \]
meaning the CDF of $Z_N$ converges pointwise to the CDF of $N(0, 1)$. 
\end{theorem}

\begin{example}[Bernoulli and binomial distributions]
\label{ex:1.7.4}
When $X_i \sim \text{Ber}(p)$, $S_N \sim \text{Binom}(N, p)$. In particular, \cref{thm:1.7.3} gives us
\[ \frac{S_N - Np}{\sqrt{Np(1 - p)}} \to N(0, 1) \text{ in distribution. } \]
The special case above is called the \underline{de Moivre-Laplace theorem}.
\end{example}

There is also a version of the CLT used for the Poisson distribution, when $p \to 0$ for the Bernoulli 
random variables: 

\begin{definition}[]
\label{def:1.7.5}
A random variable $X$ has the \underline{Poisson distribution} with parameter $\lambda > 0$, denoted 
$X \sim \text{Pois}(\lambda)$, if 
\[ P(X = k) = e^{-\lambda} \frac{\lambda^k}{k!}, \ k \in \mathbb{N}_0. \]
\end{definition}

\begin{theorem}[Poisson limit theorem]
\label{thm:1.7.6}
Consider independent random variables $X_{N, i} \si, p_{N, i}$ for $N \in \mathbb{N}$ and $1 \leq i \leq N$. 
Let 
\[ S_N = X_{N, 1} + \cdots + X_{N, N}. \]
Assume that as $N \to \infty$, 
\[ \max_{i \leq N} p_{N, i} \to 0 \text{  and  } \mathbb{E}[S_N] = \sum_{i = 1}^{N} p_{N, i} 
\to \lambda < \infty. \]
Then as $N \to \infty$, 
\[ S_N \to \text{Pois}(\lambda) \text{ in distribution. } \]
\end{theorem}

To approximate the Poisson distributions, we often have to deal with factorials. Here are a few useful tools 
for approximations:
\begin{lemma}[Stirling approximation]
\label{lem:1.7.7}
\[ n! = \sqrt{2 \pi n} \left( \frac{n}{e} \right)^n (1 + o(1)) \text{ as } n\to \infty. \]
In particular, for $X \sim \text{Pois}(\lambda)$, 
\[ P(Z = k) = \frac{e^{-\lambda}}{\sqrt{2 \pi k}} \left( \frac{e \lambda}{k} \right)^k (1 + o(1)) 
\text{ as } k \to \infty. \]
\end{lemma}

Of course, there are also non-asymptotic results:
\begin{lemma}[Bounds on the factorial]
\label{lem:1.7.8}
For any $n \in \mathbb{N}$, we have 
\[ \left( \frac{n}{e} \right)^n \leq n! \leq en \left( \frac{n}{e} \right)^n. \]
\end{lemma}

\begin{proof}
For the lower bound, we use the Taylor series for $e^x$ and drop all terms except the $n$th one, which 
gives 
\[ e^x \geq \frac{x^n}{n!}. \]
Substitute $x = n$ and rearranging gives the inequality.

For the upper bound, 
\[ \ln{(n!)} \leq \sum_{k = 1}^{n} \ln{k} \leq \int_{1}^{n} \ln{x} \ dx + \ln{n} 
= n(\ln{n} - 1) + 1 + \ln{n}. \]
Exponentiating and rearranging gives the upper bound.
\end{proof}

\begin{remark}[Gamma function]
\label{rmk:1.7.9}
The \underline{gamma function} extends the notion of the factorial to all real numbers, even to all 
complex numbers with positive real part. It is defined as 
\[ \Gamma(z) := \int_{0}^{\infty} t^{z - 1}e^{-t} \ dt. \]
Repeated integration by parts gives 
\[ \Gamma(n + 1) = n!, \ n \in \mathbb{N}_0. \]
Stirling approximation (\cref{lem:1.7.7}) is also valid for the gamma function:
\[ \Gamma(z) = \sqrt{2 \pi z} \left( \frac{z}{e} \right)^z (1 + o(1)) \text{ as } z \to \infty. \]
\end{remark}


\section{Random Matrices}


% Exercise 4.1
\section*{Exercise 4.1}
\subsection*{(a)}
This is just note true in general. We can only do this if $A$ is symmetric, as in general the left and right 
singular vectors cannot cancel out.

\subsection*{(b)}
By using the matrix form of the SVD, 
\begin{align*}
	A^{-1} 
	&= (U \Sigma V^T)^{-1} \\
	&= V^{-T} (U \Sigma)^{-1} \\
	&= V \Sigma^{-1} U^T \\
	&= \sum_{i = 1}^{n} \sigma_i^{-1} v_i u_i^T.
\end{align*}


% Exercise 4.2
\newpage
\section*{Exercise 4.2}
\subsection*{(a)}
From Definition 4.1.8, we check the criterion for a norm, namely:

For positive definiteness, we have 
\[ \lVert A \rVert_{} = \max_{x \neq 0} \frac{\lVert Ax \rVert_{2}}{\lVert x \rVert_{2}} \geq 0, \text{ and } 
= 0 \iff \lVert Ax \rVert_{2} = 0 \iff A = 0_{m \times n}. \]
For absolute homegeneity, we have (for $c \in \mathbb{R}$)
\[ \lVert cA \rVert_{} = \max_{x \neq 0} \frac{\lVert cAx \rVert_{2}}{\lVert x \rVert_{2}} 
= \frac{|c|\lVert Ax \rVert_{2}}{\lVert x \rVert_{2}} = |c|\lVert A \rVert_{}. \]
For the triangle inequality, we have 
\[ \lVert A + B \rVert_{} = \max_{x \neq 0} \frac{\lVert (A + B)x \rVert_{2}}{\lVert x \rVert_{2}} 
\leq \max_{x \neq 0} \frac{\lVert Ax \rVert_{2}}{\lVert x \rVert_{2}} 
+ \frac{\lVert Bx \rVert_{2}}{\lVert x \rVert_{2}} \leq \lVert A \rVert_{} + \lVert B \rVert_{}. \]

\subsection*{(b)}
We can write the SVDs respectively: $A = U \Sigma V^T$ and $A^T = V \Sigma^T U$. Since $\Sigma$ and $\Sigma^T$ 
have the same entries, we directly get that 
\[ \lVert A \rVert_{} = \lVert A^T \rVert_{}. \]

\subsection*{(c)}
We'll first show this: $\lVert Ax \rVert_{2} \leq \lVert A \rVert_{}\lVert x \rVert_{2}$ for all $x$. If $x = 0$ 
then the inequality is trivial. If $x \neq 0$, 
\[ \lVert A \rVert_{} = \max_{x \neq 0} \frac{\lVert Ax \rVert_{2}}{\lVert x \rVert_{2}} \geq 
\frac{\lVert Ax \rVert_{2}}{\lVert x \rVert_{2}}. \]
With this fact, we have 
\[ \lVert AB \rVert_{} = \max_{x \neq 0} \frac{\lVert ABx \rVert_{2}}{\lVert x \rVert_{2}} 
\leq \max_{x \neq 0} \frac{\lVert B \rVert_{}\lVert Ax \rVert_{2}}{\lVert x \rVert_{2}} 
= \lVert A \rVert_{} \lVert B \rVert_{}. \]

\subsection*{(d)}
Let $A$ be the matrix and $B$ be its submatrix, and let $u$ be a unit vector. We have 
\[ \lVert B \rVert_{} = \max_{\lVert u \rVert_{2} = 1} \lVert Bu \rVert_{2} = 
\max_{\lVert u \rVert_{2} = 1} \lVert \begin{bmatrix}
B \\
0 \\
\end{bmatrix} \begin{bmatrix}
u \\
0 \\
\end{bmatrix} \rVert_{2} 
\leq \max_{\lVert u \rVert_{2} = 1} \lVert A \begin{bmatrix}
u \\
0 \\
\end{bmatrix} \rVert_{2} \leq \max_{\lVert u \rVert_{2} = 1} \lVert Au \rVert_{2} = \lVert Au \rVert_{}. \]


% Exercise 4.3
\newpage
\section*{Exercise 4.3}
\subsection*{(a)}
We can write the reduced svd of $uv^T$ as 
\[ uv^T = u' \cdot \lVert u \rVert_{2}\lVert v \rVert_{2} \cdot v'^T \]
where $u', v'$ are unit vectors, hence we get the equality 
\[ \lVert uv^T \rVert_{} = \lVert u \rVert_{2}\lVert v \rVert_{2}. \]

We also have that 
\begin{align*}
	\lVert u \rVert_{2}\lVert v \rVert_{2} 
	&= \sqrt{\sum_{i = 1}^{m} u_i^2} \cdot \sqrt{\sum_{j = 1}^{n} v_j^2} \\
	&= \sqrt{\sum_{i = 1}^{m}\sum_{j = 1}^{n} (u_iv_j)^2} \\
	&= \lVert uv^T \rVert_{F}.
\end{align*}

\subsection*{(b)}
We can write the matrix $A$ as 
\[ A = \sum_{i = 1}^{n} a_i e_i e_i^T \]
where $e_i$ is the $i$th standard basis vector in $\mathbb{R}^n$ (which in this case are both the left and right 
singular vectors). To express $A$ in its SVD, we'll have to look at $a_i$: if it is negative, we'll have to 
reverse the sign, and put another negative sign on one of the singular vectors. Therefore we get
\[ \lVert A \rVert_{} = \max_{i} |a_i|. \]


% Exercise 4.4
\newpage
\section*{Exercise 4.4}
\subsection*{(a)}
The first inequality directly follows from Lemma 4.1.11:
\[ \lVert A \rVert_{} = \sigma_1 \leq \sqrt{\sigma_1^2 + \cdots + \sigma_r^2} = \lVert A \rVert_{F}. \]
For the second inequality, 
\[ \lVert A \rVert_{F} = \sqrt{\sigma_1^2 + \cdots + \sigma_r^2} \leq \sqrt{\sigma_1^2 + \cdots + \sigma_1^2} 
= \sqrt{r}\lVert A \rVert_{}. \]

\subsection*{(b)}
By Proposition 3.2.1 (b), 
\begin{align*}
	\mathbb{E}\left[ \lVert AZ \rVert_{2}^2 \right]
	&= \mathrm{tr}(\mathbb{E}\left[ AZZ^T A^T \right]) \\
	&= \mathrm{tr}(A \mathbb{E}\left[ ZZ^T \right] A^T) \\
	&= \mathrm{tr}(AA^T) \quad \text{(By isotropy)} \\
	&= \lVert A \rVert_{F}^2.
\end{align*}

\subsection*{(c)}
Let $g \sim N(0, I_n)$. We know that $g$ is isotropic. Then by part (b), 
\[ \lVert BA \rVert_{F}^2 = \mathbb{E}\left[ \lVert BAg \rVert_{2}^2 \right] 
\leq \lVert B \rVert_{}^2 \mathbb{E}\left[ \lVert Ag \rVert_{2}^2 \right] = \lVert B \rVert_{}^2 
\lVert A \rVert_{F}^2. \]
Taking square roots on both sides gives the result.

\subsection*{(d)}
In the simpler case where $A$ is diagonal, we can actually write the matrix $BA$ out explicitly: 
\[ BA = \begin{bmatrix}
a_{11}b_{11} & a_{22}b_{12} & \cdots & a_{mm}b_{1m} \\
a_{11}b_{21} & a_{22}b_{22} & \cdots & a_{mm}b_{2m} \\
\vdots & \vdots & \ddots & \vdots \\
a_{11}b_{k1} & a_{22}b_{k2} & \cdots & a_{mm}b_{km} \\
\end{bmatrix}. \]
Then we can directly get that 
\begin{align*}
	\lVert BA \rVert_{F}^2 
	&= \sum_{i = 1}^{m} a_{ii}^2 \lVert B_{:i} \rVert_{2}^2 \\
	&\leq \sum_{i = 1}^{m}a_{ii}^2 \cdot \max_{i = 1, \dots, m} \lVert B_{:i} \rVert_{2}^2 \\
	&= \lVert B \rVert_{1 \to 2}^2 \lVert A \rVert_{F}^2.
\end{align*}
Taking square roots on both sides gives the result.


% Exercise 4.5
\newpage
\section*{Exercise 4.5}
From Lemma 4.1.11, 
\[ \lVert A \rVert_{F}^2 = \sigma_1^2 + \cdots + \sigma_r^2. \]
There are two cases: 

\textbf{Case 1: $k \leq r$.} We have 
\[ k \sigma_k^2 \leq \sum_{i = 1}^{k} \sigma_1^2 \leq \lVert A \rVert_{F}^2. \]
Taking square roots on both sides gives the result.

\textbf{Case 2: $K > r$.} For $k > r$, $\sigma_k = 0$ hence the inequality holds trivially.


% Exercise 4.6
\newpage
\section*{Exercise 4.6}


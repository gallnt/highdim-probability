\section{Appetizers}

% Exercise 1
\section*{Exercise 1}
\subsection*{(a)}
\begin{align*}
	\mathbb{E}[\|Z - \mathbb{E}[Z]\|_2^2] 
	&= \mathbb{E}[\|Z\|_2^2 - 2 \langle Z, \mathbb{E}[Z] \rangle + \|\mathbb{E}[Z]\|_2^2] \\
	&= \mathbb{E}[\|Z\|_2^2] - 2 \mathbb{E}[Z]^T \mathbb{E}[Z] + \|\mathbb{E}[Z]\|_2^2 \\
	&= \mathbb{E}[\|Z\|]_2^2 - \|\mathbb{E}[Z]\|_2^2.
\end{align*}

\subsection*{(b)}
From part (a), 
\begin{align*}
	\mathbb{E}[\|Z - \mathbb{E}[Z]\|_2^2] 
	&= \mathbb{E}[\|Z\|]_2^2 - \|\mathbb{E}[Z]\|_2^2 \\
	&= \frac{1}{2}\mathbb{E}[\|Z\|_2^2] - \mathbb{E}[Z]^T \mathbb{E}[Z] + \frac{1}{2}\mathbb{E}[\|Z\|_2^2] \\
	&= \frac{1}{2} (\mathbb{E}[\|Z\|_2^2] - 2 \mathbb{E}[Z^t]\mathbb{E}[Z'] 
	+ \frac{1}{2}\mathbb{E}[\|Z'\|_2^2]) \\
	&= \frac{1}{2} (\mathbb{E}[\|Z\|_2^2] - 2 \mathbb{E}[Z^T Z'] + \mathbb{E}[\|Z'\|_2^2]) \\
	&= \frac{1}{2}\mathbb{E}[\|Z - Z'\|_2^2].
\end{align*}


% Exercise 2
\newpage
\section*{Exercise 2}
Let $\mu = \mathbb{E}[Z]$. First, notice that 
\begin{align*}
	\mathbb{E}[\|Z - a\|_2^2] - \mathbb{E}[\|Z - \mu\|_2^2] 
	&= \mathbb{E}[\|Z\|_2^2 - 2a^T Z + \|a\|_2^2 - \|Z\|_2^2 + 2 \mu^T Z - \|\mu\|_2^2] \\
	&= \|a\|_2^2 - 2(a^T - \mu^T)\mathbb{E}[Z] - \|\mu\|_2^2 \\
	&= \|a\|_2^2 - 2a^T \mu + 2\|\mu\|_2^2 - \|\mu\|_2^2 \\
	&= \|a - \mu\|_2^2.
\end{align*}
From above, minimizing $\mathbb{E}[\|Z - a\|_2^2]$ in terms of $a$ is the same as minimizing the term 
we have above as the second term does not depend on $a$. The expression above is minimized exactly at 
$a^* = \mu = \mathbb{E}[Z]$ as the quantity is always greater than or equal to 0, and reaches the value 
$0$ if and only if $a = \mu$.


% Exercise 3
\newpage
\section*{Exercise 3}
\begin{align*}
	\mathbb{E} \biggl[ \bigg\| \sum_{j = 1}^{k} Z_j \bigg\|_2^2 \biggr] 
	&= \mathbb{E}[(Z_1 + \cdots + Z_k)^T (Z_1 + \cdots + Z_k)] \\
	&= \mathbb{E} \biggl[ \sum_{j = 1}^{k} \|Z_j\|_2^2 + \sum_{i \neq j}^{} Z_i^T Z_j \biggr] \\
	&= \mathbb{E} \biggl[ \sum_{j = 1}^{k} \|Z_j\|_2^2 \biggr] 
	+ \sum_{i \neq j}^{} \mathbb{E}[Z_i]^T \mathbb{E}[Z_j] \\
	&= \mathbb{E} \biggl[ \sum_{j = 1}^{k} \|Z_j\|_2^2 \biggr] + 0 \quad\quad (\mathbb{E}[Z_i] = 0) \\
	&= \mathbb{E} \biggl[ \sum_{j = 1}^{k} \|Z_j\|_2^2 \biggr].
\end{align*}


% Exercise 4
\newpage
\section*{Exercise 4}
\subsection*{(a)}
We can consider these points as being chosen randomly at uniform from the unit ball in $n$ dimensions, i.e. 
\[ X_1, \dots, X_n \sim_{iid} \text{Unif}(B_1^n) \implies \mathbb{E}[X_i] = 0. \]
Then by exercise 3, 
\[ \mathbb{E} \biggl[ \bigg\| \sum_{i = 1}^{k} X_i \bigg\|_2^2 \biggr] 
= \sum_{i = 1}^{k} \mathbb{E}[\|X_i\|_2^2] \leq k. \]
Therefore there exists a realization $(x_1, \dots, x_n)$ for which 
\[ \bigg\| \sum_{i = 1}^{n} x_i \bigg\|_2^2 \leq k \implies 
\bigg\| \sum_{i = 1}^{n} x_i \bigg\|_2 \leq \sqrt{k}. \]

\subsection*{(b)}
We are bounding $\mathbb{E}[\|X_i\|_2^2]$ by $1$, which is a tight bound.


% Exercise 5
\newpage
\section*{Exercise 5}


% Exercise 6
\newpage
\section*{Exercise 6}
The first inequality comes as follows: we can see that 
\[ \frac{n}{k} \leq \frac{n - i}{k - i}, \quad i = 1, 2, \dots, k - 1. \]
This is because by cross multiplication
\[ n(k - i) = nk - ni \geq nk - ki = k(n - i). \]
Then
\[ \biggl( \frac{n}{k} \biggr)^k 
= \frac{n}{k} \times \frac{n}{k} \times \cdots \times \frac{n}{k} 
\leq \frac{n}{k} \times \frac{n - 1}{k - 1} \times \cdots \times \frac{n - k + 1}{1} 
= \binom{n}{k}. \]
The second inequality is trivial as $k \geq 1$.
For the third inequality, we get 
\begin{align*}
	\sum_{j = 0}^{k} \binom{n}{j} \cdot \biggl( \frac{k}{n} \biggr)^k 
	&\leq \sum_{j = 0}^{k} \binom{n}{j} \cdot \biggl( \frac{k}{n} \biggr)^j \quad (k/n \leq 1)\\
	&\leq \sum_{j = 0}^{n} \binom{n}{j} \cdot \biggl( \frac{k}{n} \biggr)^j \quad (k/n \leq 1)\\
	&= \biggl( 1 + \frac{k}{n} \biggr)^n \quad \text{(Binomial Theorem)} \\
	&< e^k.
\end{align*}


% Exercise 7
\newpage
\section*{Exercise 7}
Assume $n$ is large so that the $5/n$ radius near the surface is valid. The inner ball has radius 
$\frac{1}{2} - \frac{5}{n}$. Then the volume of the inner ball is $(\frac{1}{2} - \frac{5}{n})^n$ times 
the volums of the outer unit ball. In particular, as $n \to \infty$, 
\[ \biggl( \frac{1}{2} - \frac{5}{n} \biggr)^n 
= \biggl( \frac{1}{2} \biggr)^n \biggl( 1 - \frac{10}{n} \biggr)^n \to 0. \]
This means that most of the points will be concentrated towards the surface of the $n$-dimensional ball.


% Exercise 8
\newpage
\section*{Exercise 8}
Let $X \sim \text{Unif}(B_1^n)$. Then the pdf of $X$ is 
\[ f_X(x) = \frac{1}{\text{Vol}(B_1^n)}, \ x \in B_1^n. \]
Now let's consider the random variable $\|X\|_2$, i.e. the 2-norm of the random vector. Since 
the random vector is distributed uniformly in the $n$-dimensional ball, we can define its CDF as a 
function of the radius $r$: 
\[ F_{\|X\|_2}(r) = P(\|X\|_2 \leq r) = r^n. \]
Correspondingly, we can find the PDF by just taking the derivative of the CDF: 
\[ f_{\|X\|_2}(r) = nr^{n - 1}, \ 0 \leq r \leq 1. \]
Then we can directly get that 
\[ \mathbb{E}[\|X\|_2] = \int_{0}^{1} r \cdot nr^{n - 1} \ dr 
= n \cdot \biggl[ \frac{r^{n + 1}}{n + 1} \biggr]_0^1 = \frac{n}{n + 1}. \]


